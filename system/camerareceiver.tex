The \CameraReceiver{} abstract class defines the base structure of the mechanism that will acquire and handle
data from the camera sensors. It implements two \RD{} interfaces. The first one is the \iUpdateable{} interface,
which describes classes that have an \texttt{up\-date} method. The main loop of an \RD{} application calls 
\texttt{up\-date} on all the objects that have been registered as updatable. For the camera receiver this means 
that the \texttt{up\-date} method can be used to grab an image from a camera on every loop.

The second interface used by the \CameraReceiver{} class is the \iDataRecordHandler{}. This interface is part
of \RD{}'s intra-application communication system. It is used to define a class that can handle \iDataRecord{} 
objects being sent from another application. A class that implements this interface has a 
\texttt{han\-dle\-Da\-ta\-Re\-cord} method that handles the incoming data record objects. The camera receiver
can use this method to receive records containing image data.

Table \ref{camerareceivermethods} lists the two abstract methods declared in the \CameraReceiver{} class.
These methods are used to start the image receiving process and query the receiver if it has image data 
available to be accessed or retrieved. 

\begin{table}[ht]
\caption{Public methods in the \CameraReceiver{} class }
\begin{center}
\begin{tabular}{| l |}
	\hline 
	\multicolumn{1}{| c |}{\CameraReceiver{}} \\
	\hline \hline
	\texttt{startReceiving} \\
	\texttt{isDataAvailable} \\
	\hline
\end{tabular}
\end{center}
\label{camerareceivermethods}
\end{table}