The \ColorPacketHandler{} class is a subtype of \SafePacketHandler{}, a class part of the receiving end in 
the IRCP communication system. The \ColorReceiver{} object creates a color packet handler to manage the 
packets sent over the network that correspond to image data from a color camera. Therefore, the color
packet handler needs to know how the color information is encoded in the incoming packet. 

In order to manage a large number of sent packets, some of which can get lost or arrive out of sequence, 
the color packet handler implementation uses a \PacketOrganizer{} object to organize the incoming 
data (see Section \ref{packetorganizer}). The packet organizer takes the received packets, which 
consist of subparts of the whole image, and informs the color packet handler once there are images ready
to be retrieved.

Table \ref{colorpackethandlermethods} contains the method that the color packet handler overrides 
from the safe packet handler class. 

\begin{table}[ht]
\caption{Public methods in the \ColorPacketHandler{}  class}
\begin{center}
\begin{tabular}{| l |}
	\hline 
	\multicolumn{1}{| c |}{\ColorPacketHandler{}} \\
	\multicolumn{1}{| c |}{{\small extends \SafePacketHandler{}}} \\
	\hline \hline
	\texttt{safeHandle} \\
	\hline
\end{tabular}
\end{center}
\label{colorpackethandlermethods}
\end{table}