The \ColorReceiver{} class extends \CameraReceiver{} to provide the mechanism that interfaces with a color 
camera. It allows the user to acquire images directly from a Firewire camera or to receive the color images via a 
network transfer. Image acquisition from the camera is achieved using the libdc1394 API, a library of functions to 
control Firewire cameras that conform to the 1394-based Digital Camera Specifications \cite{libdc1394}. The 
network transfer is achieved through IRCP, the network communication protocol used by \RD{} \cite{IRCP}.
Table \ref{colorreceivermethods} lists the methods that the \ColorReceiver{} class adds to the camera receiver 
representation.

\begin{table}[ht]
\caption{Public methods in the \ColorReceiver{} class }
\begin{center}
\begin{tabular}{| l |}
	\hline 
	\multicolumn{1}{| c |}{\ColorReceiver{}} \\
	\multicolumn{1}{| c |}{{\small \texttt{extends} \CameraReceiver{}}} \\
	\hline \hline
	\texttt{getImage} \\
	\texttt{handleDataRecord} \\
	\texttt{update} \\
	\hline
\end{tabular}
\end{center}
\label{colorreceivermethods}
\end{table}

Upon initialization, the color receiver first tries to connect to a Firewire camera. The \texttt{up\-date} method
is in charge of acquiring the images using the \DCJavaAcquire{} methods. If the color receiver does not find a 
camera or if it fails when attempting to connect to one, it will proceed to start the IRCP network connection by 
setting up a \ColorPacketHandler{} (the user has the option of forcing this network connection). As discussed in 
Section \ref{camerareceiver}, the images received by the packet handler are sent to the color receiver through 
the \iDataRecordHandler{} interface. The color receiver calls the \texttt{han\-dle\-Da\-ta\-Re\-cord} method to 
handle the incoming data record objects.

The \ColorReceiver{} class is flexible with the type of image it can handle. The image type is specified by a 
set of public static variables that the user can modify. Table \ref{colorreceivervariables} contains the list of all 
the variables available to the user. When running on libdc1394 mode the user can change the size of the image 
captured from the camera. When running on network mode the user can modify the image size, pixel depth, and 
color model according to what is being transferred over the network. 

\begin{table}[ht]
\caption{User-modifiable static variables in the \ColorReceiver{} class}
\begin{center}
\begin{tabular}{| l | l |}
	\multicolumn{2}{c}{\ColorReceiver{}} \\
	\hline
	Variable & Description \\
	\hline \hline
	\texttt{WIDTH} 						& The image width \\
	\texttt{HEIGHT} 					& The image height \\
	\texttt{DEPTH} 						& The image pixel depth \\
	\texttt{COLOR\_MODEL} 				& The image color model \\
	\texttt{CAMERA\_GUID} 				& The camera GUID \\
	\texttt{DC1394\_FRAME\_RATE} 		& The camera capturing frame rate \\
	\texttt{DC1394\_ISO\_SPEED} 			& The camera ISO speed \\
	\texttt{UNDISTORT} 				& Flag to undistort the image \\
	\texttt{COMPRESSED\_IMAGE} 		& Flag if image is compressed \\
	\texttt{STREAM\_CAPACITY} 			& The receiver image stream size \\
	\texttt{VISION\_MINOR\_TYPE} 		& The IRCP minor type \\
	\texttt{PACKET\_HANDLER\_NAME} 	& The packet handler name \\
	\texttt{PACKET\_HANDLER\_KEY} 				& The packet handler key \\
	\hline
\end{tabular}
\end{center}
\label{colorreceivervariables}
\end{table}

Besides handling a real time image stream, the color receiver can store a user-defined number of last seen 
images in memory. The mechanism that processes the image stream works like a First-In-First-Out queue: 
fresh images are stored at the tail of the stream while old images are retrieved from the head of the stream. 
The size of the stream determines the number of images that can be saved in memory. This system allows
to synchronize two or more cameras by finding in their streams the images with matching (or closest) 
timestamp. The \ImageStream{} class contains the implementation of this mechanism (see Section 
\ref{imagestream}).
