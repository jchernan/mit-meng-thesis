The \ImageBuffer{} class represents an image in the \RD{} environment. An instance of this class is defined 
by four fields. The first three are the image's width, height, and pixel depth. The fourth field can be either
the image's color model or its number of channels.\footnote{Not all images require a color model. An 
example is the depth image acquired by a range sensor.} 
The different possible combinations of values for these fields allow a flexible description of an image inside
the programming environment. Tables \ref{imagebufferdepth} and \ref{imagebuffercolormodel} show the 
available values for the depth and color model fields, as described by the \ImageBufferDepth{} and 
\ImageBufferColorModel{} enum types.

\begin{table}[ht]
\caption{Pixel depth values in the \ImageBufferDepth{} enum}
\begin{center}
\begin{tabular}{| l | l |}
	\multicolumn{2}{c}{\ImageBufferDepth{}} \\
	\hline 
	Depth 			& Description \\
	\hline \hline
	\texttt{BYTE} 		& Unsigned 8-bit integer \\
	\texttt{SHORT}	 	& Unsigned 16-bit integer \\
	\texttt{FLOAT}	 	& 32-bit floating-point single precision \\
	\texttt{DOUBLE} 	& 64-bit floating-point double-precision \\
	\hline
\end{tabular}
\end{center}
\label{imagebufferdepth}
\end{table}

\begin{table}[ht]
\caption{Color model values in the \ImageBufferColorModel{} enum}
\begin{center}
\begin{tabular}{| l | l |}
	\multicolumn{2}{c}{\ImageBufferColorModel{}} \\
	\hline 
	Color Model 		& Description \\
	\hline \hline
	\texttt{BGR} 		& Blue-Green-Red channels \\
	\texttt{BGRA} 		& Blue-Green-Red-Alpha channels \\
	\texttt{RGB}	 	& Red-Green-Blue channels \\
	\texttt{RGBA}	 	& Red-Green-Blue-Alpha channels \\
	\texttt{GRAY}	 	& Grayscale channel \\
	\hline
\end{tabular}
\end{center}
\label{imagebuffercolormodel}
\end{table}

In an image buffer, the image data is stored in an underlying \ByteBuffer{} object. 
Part of the flexibility of the \ImageBuffer{} class is attributed to the versatility of a byte buffer with
respect to the data it can hold. Furthermore, the byte buffer within an image buffer is created to be 
direct. A direct byte buffer in Java allocates its space in native system memory, outside of the Java 
virtual machine's memory heap. This allows other programs outside Java to use native code to access
the same allocated memory space. An example of the advantage of using direct buffers is the 
implementation of an interface between \RD{} and the OpenCV computer vision open source library.
This interface enables performing OpenCV functions directly onto the Java images through native code.

Pixels in an image buffer are stored row by row with interleaved channels, starting at the top-left pixel. 
The methods to access and manipulate the image data are specific to the image's pixel depth. For 
example, the methods of the form \texttt{get\-Byte\-Pix\-el} are used to access pixels in an image of depth 
\texttt{BYTE}. The different versions of the same getter method trade off convenience for performance. 
Getter methods that accept an array as input speed up a loop through the image's pixels by
avoiding the overhead of creating the output array on every call.
