The \ImageStream{} class extends \Stream{} to represent a stream of images. An \ImageBuffer{} (see Section
\ref{imagebuffer}) is used as the buffer object to holds image data. Consequently, the constructor of an 
\ImageStream{} requires as argument the width, height, pixel depth, and color model information that 
describes the type of image it needs to handle. 

Given the specified capacity for the stream, an \ImageStream{} object will instantiate that number of image 
buffers, and will only use those during the lifetime of the object. The advantage of this design becomes clear 
when considering that the \ImageBuffer{} class performs the expensive operation of allocating space in native 
memory. Instead of reallocating that space multiple times, the buffer is created once and reused afterwards.

An image stream behaves like a First-In-First-Out queue. The images recently acquired by a camera are 
added at the tail of the stream using the \texttt{add} method listed in Table \ref{imagestreammethods}. When 
the stream has reached maximum capacity, each call to \texttt{add} polls an old image from the head of the 
stream before adding the new image at the tail. In this way, at any given time, the stream stores a number of 
consecutive images equal to the capacity of the stream.

\begin{table}[ht]
\caption{Public methods in the \ImageStream{} class}
\begin{center}
\begin{tabular}{| l |}
	\hline 
	\multicolumn{1}{| c |}{\ImageStream{}} \\
	\multicolumn{1}{| c |}{{\small \texttt{extends} \Stream{}}} \\
	\hline \hline
	\texttt{add} \\
	\texttt{peekLast} \\
	\hline
\end{tabular}
\end{center}
\label{imagestreammethods}
\end{table}

The \ImageStream{} class provides the \texttt{peek\-Last} method to access the most recently added image 
from the stream. Given that the stream holds a history of past images, the user has realtime access to the 
image data by using the \texttt{peek\-Last} method. A second version of the \texttt{peek\-Last} method takes
as input the desired timestamp for the retrieved image, and searches in the stream for the image with 
matching (or closest) timestamp. This method is used by the \ColorReceiver{} and \SwissRangerReceiver{} 
classes to get image data given a specified timestamp (see Sections \ref{colorreceiver} and 
\ref{swissrangerreceiver}). 