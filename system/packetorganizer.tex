When sending packets through the network, packets might arrive at the receiver end out of order, or they 
might get dropped and not be received at all. Furthermore, the sender might divide the packets into smaller 
parts, and send those to the receiver in the form of subpackets. The receiver must sort the subpackets and 
know when the whole packet has been received. The receiver should also decide when to stop waiting for a 
dropped subpacket.

The \PacketOrganizer{} class provides a mechanism to handle these situations. When creating a packet 
organizer, the user decides how many buffers will be collecting subpackets at the same time, each buffer
being assigned to a different packet. A received subpacket must contain header information about its position 
in the subpacket stream, its length, the timestamp of the packet to which it belongs, and the total number of 
subpackets that form this packet.

\begin{table}[ht]
\caption{Public methods in the \PacketOrganizer{} class}
\begin{center}
\begin{tabular}{| l |}
	\hline 
	\multicolumn{1}{| c |}{\PacketOrganizer{}} \\
	\hline \hline
	\texttt{put} \\
	\texttt{getRecord} \\
	\texttt{getBuffer} \\
	\texttt{ready} \\
	\texttt{releaseBuffer} \\
	\hline
\end{tabular}
\end{center}
\label{packetorganizermethods}
\end{table}

Table \ref{packetorganizermethods} lists the methods available in the \PacketOrganizer{} class. The 
\texttt{put} method is used to insert a received subpacket into its corresponding packet buffer. The 
\texttt{read\-y} method indicates if a packet buffer has received all the expected subpackets. Once a packet
buffer is ready, the methods \texttt{get\-Re\-cord} and \texttt{get\-Buff\-er} are used to retrieve the data from
that buffer. The user should copy the data out of the buffer because buffers are reused during the lifetime of 
the packet organizer. Once the data has been retrieved, the method \texttt{re\-lease\-Buff\-er} tells the 
packet organized that the buffer can be used to collect subpackets from another incoming packet.

