The \Stream{} abstract class represents a stream of buffer objects. A buffer object can be any mutable object 
used to store data. Since the \Stream{} class is defined using a generic type declaration, it allows the user
decide what will the buffer object be in an specific stream implementation. 

When deciding on a buffer object, the user has to make sure it can be created and manipulated using the 
abstract methods listed in Table \ref{streamprotectedmethods}. These methods should be implemented by 
the user in the \Stream{}'s subclasses, where \texttt{T} is substituted by the type of the buffer object. The 
\texttt{buff\-er} argument in the \texttt{write\-Buff\-er} and \texttt{read\-Buff\-er} methods represents the buffer
where data is going to be written or from where data is going to be read, respectively. Conversely, the 
\texttt{da\-ta} argument represents the data that will be written into the buffer or where the data read from the 
buffer will be copied into, respectively.

\begin{table}[ht]
\caption{Abstract methods that define the buffer object in the \Stream{} class}
\begin{center}
\begin{tabular}{| l |}
	\hline 
	\multicolumn{1}{| c |}{\Stream{}} \\
	\hline \hline
	\texttt{T createBuffer()} \\
	\texttt{writeBuffer(T buffer, T data)} \\
	\texttt{readBuffer(T buffer, T data)} \\
	\hline
\end{tabular}
\end{center}
\label{streamprotectedmethods}
\end{table}

The \Stream{} object is created with a finite capacity that remains constant during the lifetime of the object. 
Therefore, it creates a finite amount of buffer objects that are instantiated only once and always reused 
afterwards. The \Stream{} class' design allows to operate similar to a First-In-First-Out queue, with the 
incoming data being added at the tail of the stream and the old data being polled from the head of the stream. 
Furthermore, the data is available to be retrieved only when all the buffers in the stream have been filled, 
ensuring in that way that all buffers contain meaningful continuous data that the user can use at any given 
time.

The \Stream{} class contains the implementation of the public methods listed in Table \ref{streammethods}. 
It also provides some non-public helper methods that are used by the subclasses to add, poll, and peek the 
stream. The user must declare in the subclasses the formal methods to access the data from the stream 
according to the required design. 

\begin{table}[ht]
\caption{Public methods in the \Stream{} class}
\begin{center}
\begin{tabular}{| l |}
	\hline 
	\multicolumn{1}{| c |}{\Stream{}} \\
	\hline \hline
	\texttt{isDataAvailable} \\
	\texttt{capacity} \\
	\texttt{get} \\
	\texttt{clear} \\
	\hline
\end{tabular}
\end{center}
\label{streammethods}
\end{table}