The \SwissRangerReceiver{} class extends \CameraReceiver{} to provide the mechanism that interfaces with a 
SwissRanger camera. It allows the user to acquire images directly from a SwissRanger or to receive depth, 
amplitude, x-coordinate, y-coordinate, and confidence images via a network transfer. Image acquisition from 
the camera is achieved using the SwissRanger driver API that is provided by Mesa Imaging. The network 
transfer is achieved through IRCP. Table \ref{swissrangerreceivermethods} lists the methods of the 
SwissRanger receiver class.

\begin{table}[ht]
\caption{Public methods in the \SwissRangerReceiver{} class}
\begin{center}
\begin{tabular}{| l |}
	\hline 
	\multicolumn{1}{| c |}{\SwissRangerReceiver{}} \\
	\multicolumn{1}{| c |}{{\small \texttt{extends} \CameraReceiver{}}} \\
	\hline \hline
	\texttt{getDepth} \\
	\texttt{getAmplitude} \\
	\texttt{getX} \\
	\texttt{getY} \\
	\texttt{getConfidenceMap} \\
	\texttt{handleDataRecord} \\
	\texttt{update} \\
	\hline
\end{tabular}
\end{center}
\label{swissrangerreceivermethods}
\end{table}

Similar to the color receiver, the SwissRanger receiver first tries to connect to a Swiss\-Ranger camera. The 
\texttt{up\-date} method, called on each system loop, grabs the images from the camera using the 
\SwissRangerJavaAcquire{} methods. If the receiver fails to connect to a camera it proceeds to start the IRCP 
network connection by setting up a \SwissRangerPacketHandler{} (again, the user has the option of forcing a 
network connection). The images received by the packet handler are sent to the SwissRanger receiver, which 
calls the \texttt{han\-dle\-Da\-ta\-Re\-cord} method to handle the incoming data record objects.

The \SwissRangerReceiver{} class is flexible with the type of image it can handle and it provides a set of 
public static variables that the user can modify. However, in practice these variables are not changed 
because the SwissRanger sensor itself does not provide this flexibility. Table \ref{swissrangerreceivervariables} 
contains the list of all the variables available to the user. As this table shows, the user has the option of 
thresholding the depth image. This thresholding is performed based on depth, amplitude, and confidence map
values. 

\begin{table}[ht]
\caption{User-modifiable static variables in the \SwissRangerReceiver{} class}
\begin{center}
\begin{tabular}{| l | l |}
	\multicolumn{2}{c}{\SwissRangerReceiver{}} \\
	\hline
	Variable & Description \\
	\hline \hline
	\texttt{WIDTH} 								& The image width \\
	\texttt{HEIGHT} 							& The image height \\
	\texttt{DEPTH} 								& The image pixel depth \\
	\texttt{NUMBER\_OF\_CHANNELS} 				& The image number of channels \\
	\texttt{MODULATION\_FREQUENCY} 			& The modulation frequency of the camera \\
	\texttt{INITIAL\_DEPTH\_HIGH\_THRESHOLD} 	& The initial depth high threshold \\
	\texttt{INITIAL\_DEPTH\_LOW\_THRESHOLD} 	& The initial depth low threshold \\
	\texttt{INITIAL\_AMPLITUDE\_THRESHOLD} 		& The initial amplitude threshold \\
	\texttt{INITIAL\_CONFIDENCE\_THRESHOLD} 	& The initial confidence threshold \\
	\texttt{MAX\_AMPLITUDE\_THRESHOLD} 		& The maximum threshold for the amplitude \\
	\texttt{MAX\_CONFIDENCE\_THRESHOLD} 		& The maximum threshold for the confidence \\
	\texttt{UNDISTORT} 						& Flag to undistort the image \\
	\texttt{THRESHOLD} 						& Flag to threshold the depth image \\
	\texttt{STREAM\_CAPACITY} 					& The receiver image stream size \\
	\texttt{VISION\_MINOR\_TYPE} 				& The IRCP minor type \\
	\texttt{PACKET\_HANDLER\_NAME} 			& The packet handler name \\
	\texttt{PACKET\_HANDLER\_KEY} 				& The packet handler key \\
	\hline
\end{tabular}
\end{center}
\label{swissrangerreceivervariables}
\end{table}

The Swiss\-Ranger receiver handles the different image streams using the First-In-First-Out queue mechanism 
used by the color receiver: fresh images are stored at the tail of the stream while old images are retrieved 
from the head of the stream. This allows the receiver to store a user-defined number of last seen images in 
memory and to synchronize two or more cameras by locating in their streams the images with matching (or 
closest) timestamp. Like in the \ColorReceiver{} class, the implementation of this mechanism is provided by an 
\ImageStream{} object (see Section \ref{imagestream}).
